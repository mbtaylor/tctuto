\documentclass{article}

\usepackage{graphicx}
\usepackage{color}
\usepackage{xcolor}
\usepackage{url}
% \usepackage{parskip}

\setlength{\unitlength}{1cm}


% Fixes problem with includegraphics images screwing up colours on their
% page in the output PDF.  I have no idea *how* it fixes it mind.
\pdfpageattr {/Group << /S /Transparency /I true /CS /DeviceRGB>>}

\pagestyle{plain}

\setlength{\topmargin}{-3cm}
\setlength{\textheight}{25.5cm}
\setlength{\oddsidemargin}{-2.0cm}
\setlength{\evensidemargin}{-2.0cm}
\setlength{\textwidth}{20cm}

\definecolor{brown}{rgb}{0.5,0.25,0.0}
\definecolor{grey}{rgb}{0.6,0.6,0.6}

\newcommand{\buttimg}[1] % filename
           {\mbox{\vtop{\vskip-2ex\hbox{\includegraphics[height=3ex]
                                                        {icons/#1}}}}}
\newcommand{\winfig}[2]  % includegraphics-options, filename
           {\vspace*{-0.5cm}
            \hspace*{0.5cm}\mbox{\vtop{\hbox{\includegraphics[#1]
                                                             {figures/#2}}}}}
\newcommand{\lab}[1]{{\bf #1}}
\newcommand{\ma}[2]{\buttimg{#1}~\lab{#2}}
\newcommand{\mb}[3]{\buttimg{#1}~\lab{#2}$\mid$\lab{#3}}
\newcommand{\entry}[2]{\lab{#1}: ``{\tt #2}''}
\newcommand{\turl}[1]{{\color{blue}\url{#1}}}
\newcommand{\paragap}{\vspace{1.5ex}}

\begin{document}
\raggedright

\title{Tutorial: Exploring Gaia data with TOPCAT and STILTS}
\author{Mark Taylor,\\
        University of Bristol,\\
        m.b.taylor@bristol.ac.uk}
\date{\vspace*{-4ex}}  %  git date used in footer instead

\maketitle

\begin{description}
  \item[TOPCAT:] \turl{http://www.starlink.ac.uk/topcat/}
                 {\em (version 4.7 or later recommended)}
  \item[STILTS:] \turl{http://www.starlink.ac.uk/stilts/}
                 {\em (version 3.2 or later recommended)}
  \item[Mailing list:] {\tt topcat-user@jiscmail.ac.uk}
  \item[Version:] {\tt \input{version}}
\end{description}

\tableofcontents

\vspace{3ex}
This tutorial uses data from Gaia Early Data Release 3 (EDR3) \cite{edr3}
to lead you through some of the capabilities of TOPCAT and STILTS.
For best results, you should have the manuals to hand:
\url{http://www.starlink.ac.uk/topcat/sun253/} and
\url{http://www.starlink.ac.uk/stilts/sun256/}.

\newpage

\section{Cluster identification \#1: Messier 4 in proper motion space}
\label{sec:m4}

\begin{minipage}[t]{11cm}
In this example we will determine the mean parallax of
the stars in the globular cluster Messier 4 (M4, or NGC 6121).

\subsection{Acquire Gaia data in the M4 region}
\label{sec:m4-cone}

  \raggedright
  \begin{enumerate}
  \item Start TOPCAT.
  \item Open the \mb{CONE_DIALOG.png}{VO}{Cone Search} window \\
        (i.e.\ use the \lab{Cone Search} submenu of the \lab{VO} menu
         in the main topcat window).
  \item Fill in \entry{Keywords}{gaia edr3}, and hit \lab{Find Services}.
  \item There are a few options, that should mostly give similar results.
        The one with Short Name ``{\tt eDR3 lite Cone}'' is a good choice.
        Select it by clicking on its row, and the
        partial URL of the service appears in the \lab{Cone URL} field.
  \item Fill in \entry{Object Name}{M4}, then hit \lab{Resolve} to fill in
        sky position fields.
  \item \entry{Radius}{0.3} (degrees)
  \item Hit \lab{OK};
        new table is loaded into topcat main control window
        with about 50\,000 rows.
        If the download is too slow, cancel and try with \entry{Radius}{0.1},
        which retrieves about 20\,000 rows,
        and maybe \entry{Verbosity}{1},
        which reduces the number of columns requested.
  \item Use the \mb{skyplot_button.png}{Graphics}{Sky Plot} window
        to see the positions on the sky.
  \item Play with the plot.  Note overdense regions are coloured darker.
        Use the options in the \lab{Form} tab to change marker size,
        colour and shading.
        Practice navigation: use mouse drag and wheel (or CTRL-drag).
        Click the little \buttimg{navig_help.png} button at bottom left
        for navigation help;
        note navigation details are different for different plot types.
  \end{enumerate}
\end{minipage}
\begin{minipage}[t]{8cm}
  \winfig{width=8cm}{m4_cone.png}
\end{minipage}

\vspace{1cm}
\subsection{Identify comoving cluster}
\label{sec:blob}
\label{sec:m4-pm}

\begin{minipage}[t]{11cm}
  \raggedright
  \begin{enumerate}
  \item Plot sources in proper motion space: \\
        \mb{planeplot_button.png}{Graphics}{Plane Plot} window, \\
        \entry{X}{pmra}, \entry{Y}{pmdec}
  \item Note overdensity far from (0,0);
        use mouse to navigate 
  \item Graphically select this comoving cluster as new Subset:
        \mb{blob_button.png}{Subsets}{Draw Subset Region} button, \\
        drag mouse around cluster, hit \buttimg{unblob_button} button again
  \item A \lab{New Subset} dialogue pops up:
        fill in \entry{New Subset Name}{comoving}, \lab{Add Subset}
  \item Look in \lab{Subsets} tab of plot window;
        turn \lab{All} and \lab{comoving} subsets off/on
  \end{enumerate}
\end{minipage}
\begin{minipage}[t]{8cm}
  \begin{picture}(8,0)
  \put(0,-6){\winfig{width=8cm}{m4_pmplot.png}}
  \end{picture}
\end{minipage}

\begin{minipage}[t]{11cm}
\subsection{Manage Subsets}
  \raggedright
  \begin{enumerate}
  \item Open the \mb{subsets_button.png}{Views}{Row Subsets} window
  \item See the new \lab{comoving} subset
  \item Create an additional subset that contains the background objects,
        i.e.\ those not in the comoving set:
        click to select the \lab{comoving} row,
        then select the \mb{invert_button}{Subsets}{Invert Subset} action.
  \item A new subset \lab{not\_comoving} will appear;
        if you like you can rename it ``\lab{background}'' by double-clicking
        in the \lab{Name} field
  \end{enumerate}

\subsection{Examine cluster members}

  \raggedright
  \begin{enumerate}
  \item Go back to the \mb{skyplot_button.png}{Graphics}{Sky Plot}
        from section \ref{sec:m4-cone} (or open a new one)
  \item In \lab{Subsets} tab turn different subsets on/off
        using checkboxes
  \item Plot the proper motions.
        In the \lab{Form} tab, use the \ma{Plus1.png}{Forms} menu
        and select the \ma{ADD_FORM_VECTOR.png}{Add SkyVector} item,
        with \entry{Delta Lon(*)}{pmra},
             \entry{Delta Lat}{pmdec}.
        The arrows will initially be much too long (units of degrees);
        you will have to set \entry{Unit}{scaled} (auto-scaling),
        and optionally adjust to taste with the \lab{Scale} slider.
        Zoom in so you can see some individual objects.
        All the cluster objects have similar proper motions,
        non-cluster ones have various directions, or none (no measured P.M.).
  \end{enumerate}

\subsection{Determine parallax}
\label{sec:histo-mean}

  \raggedright
  \begin{enumerate}
  \item Plot histogram of parallaxes: \\
        \mb{histo_button.png}{Graphics}{Histogram Plot}, \\
        \entry{X}{parallax}
  \item In \lab{Subsets} tab, make sure only subsets \lab{comoving} and
        \lab{not\_comoving}, not \lab{All}, are plotted
  \item Normalise histograms to the same height:
        click on the \ma{histobars.png}{Bins} control in the left-hand panel,
        select the \lab{General} tab, and set
        \lab{Normalise} to \lab{maximum}.
        Return to \ma{histoplot.png}{histogram layer} control
  \item Zoom in horizontally (mouse below X axis, use drag and wheel/CTRL-drag)
        to read off the comoving peak/average parallax.
        Note the comoving (cluster) distribution is different from the
        background sample.
  \item For a more accurate result, fit Gaussian to data:
        \ma{Plus1.png}{Forms} menu in \lab{Form} tab,
        \ma{ADD_FORM_GAUSSIAN}{Gaussian} option.
        Then scroll to bottom of Gaussian layer description in \lab{Form} tab,
        select \entry{Subset}{comoving},
        and read off parallax \lab{Mean} and \lab{S.D.}
  \item Invert mean parallax get distance to cluster
        (1000/parallax in mas = distance in parsec).
  \item To do this without plotting, you can read off the mean and S.D.\ for
        parallax in the \mb{stats_button.png}{Views}{Column Statistics} window
        for the \lab{comoving} subset.
  \end{enumerate}

  {\bf Note: careful when inverting parallaxes!}\\
  In general $r = 1/\varpi$ is {\em not reliable\/}
  because of errors.
  It's OK here because we are averaging over many measurements
  with smallish errors.
  Rule of thumb for single measurements:
  if $\varpi/\sigma_\varpi>5$ it's probably OK.
  See Luri et al.\ 2018 \cite{luri} for full discussion.
\end{minipage}
\begin{minipage}[t]{8cm}
  \winfig{width=8cm}{m4_subsets.png}
  \\[0.5cm]

  \winfig{width=8cm}{m4-vector.png}
  \\[0.5cm]

  \winfig{width=8cm}{m4-parallax.png}
\end{minipage}

\newpage

\section{Cluster identification \#2: Hyades in 3-D velocity space}

This example locates the Hyades open cluster in 3-dimensional velocity space,
using Gaia's proper motion and radial velocity observations.
We can't start this time by making a positional query (cone search),
since the Hyades is very delocalised on the sky, because it's so close,
so a cone would contain way too many sources.
So we need to make a more sophisticated query using TAP.

\subsection{Locate Gaia TAP service}

\begin{minipage}[t]{11cm}
  \raggedright

  \begin{enumerate}
  \item Open the TAP window: \\
        \mb{tap_button.png}{VO}{Table Access Protocol (TAP) Query}
  \item Fill in \entry{Keywords}{gaia} and hit \lab{Find Services} button
  \item There are several services with Gaia data in various forms;
        \lab{GAIA} (ESA) or \lab{ARI-Gaia} (Heidelberg) are good choices.
        The service URL appears in the field at the bottom of the window.
  \item Hit the \lab{Use Service} button at the bottom
  \end{enumerate}

  \subsection{Explore the TAP service}

  Use the TAP window to explore the tables that are present and
  their metadata.
  \begin{enumerate}
  \item Browse the table list on the left,
        The tables in the {\tt gaiaedr3} schema
        are the ones with Gaia EDR3 data.
  \item Select table {\tt gaiaedr3.gaia\_source} and look at
        \lab{Table} and \lab{Columns} tabs,
        to see information about available columns.
  \item Look in the \lab{Service} tab to see information about the service
  \item Look in the \lab{Hints} tab for a very basic ADQL cheat sheet
  \item Type in to the bottom panel some very simple ADQL:\\
        ``{\tt SELECT TOP 10 ra, dec FROM gaiaedr3.gaia\_source}''
  \item Note that syntax errors (including partial or misspelt tables/columns)
        are highlighted in red.
  \item Hit \lab{Run Query} to run the query;
        if successful a new table is loaded
  \end{enumerate}

  \subsection{Acquire astrometric data}

  In the TAP window, execute this query:
  \begin{verbatim}
    SELECT ra, dec, pmra, pmdec, parallax,
           dr2_radial_velocity, bp_rp,
           phot_g_mean_mag + 5*log10(parallax/100) as g_abs
    FROM gaiaedr3.gaia_source
    WHERE parallax > 15
    AND parallax_over_error > 5
    AND dr2_radial_velocity IS NOT NULL
  \end{verbatim}
  \vspace*{-2ex}
  The result should contain about 25\,000 rows.
\end{minipage}
\begin{minipage}[t]{8cm}
  \winfig{width=8cm}{hy_tapsel.png}

  \vspace*{1cm}

  \winfig{width=8cm}{hy_tapex.png}
\end{minipage}
\paragap

The query is for all the nearby sources
(nominally within 1000/15 $\approx$ 66\,parsec)
with observed radial velocities
(only about 7 million DR2/EDR3 sources have RV)
and good determinations of parallax.
The fact that parallax error is $\leq$20\% means that it's OK
to invert parallax to calculate distance and absolute magnitude.
We are retrieving all the basic astrometric parameters and some
photometry.
\paragap

The Hyades should be in there somewhere.  Can we find them?


\newpage

\subsection{Calculate 3-d velocity components}

We have the astrometric quantities measured by Gaia,
which contain full phase space information,
but need transformations to yield Cartesian position/velocity coordinates.
TOPCAT's {\em expression language\/} can help.
\vspace*{3ex}

\begin{minipage}[t]{11cm}
  \raggedright
  \begin{enumerate}
  \item Open the \mb{fx_button}{Help}{Available Functions} browser to see
        what functions TOPCAT provides (they are listed in the manual too).
  \item Look under the \lab{Gaia} entry to see astrometry-specific items
  \item We will use the {\tt astromUVW} and maybe {\tt icrsToGal}
        and {\tt astromXYZ} functions.
        The \lab{Examples} items in the function documentation are useful;
        for use with the {\tt gaia\_source} catalogue,
        you can often just cut and paste, though note
        ``{\tt radial\_velocity}'' may need to be changed to
        ``{\tt dr2\_radial\_velocity}''.
  \item Open the \mb{cols_button.png}{Views}{Column Info} window
        and choose the \mb{plus_button.png}{Columns}{New Synthetic Column}
        action
  \item Create a new column giving Cartesian velocity components:
        \entry{Name}{uvw}, \entry{Units}{km/s} and for \lab{Expression}:
        \begin{verbatim}
  astromUVW(array(ra, dec, parallax, pmra, pmdec, dr2_radial_velocity))
        \end{verbatim}
        \vspace*{-0.5cm}
        That calculates velocities along ICRS axes.
        If you want it in Galactic coordinates, wrap the whole expression
        in the {\tt icrsToGal(...)} function.
  \item Look at the new column in the \mb{browser1.png}{Views}{Table Data}
        window (scroll all the way to the right).
        It is a 3-element array; you can access the array elements
        using expressions {\tt uvw[0]}, {\tt uvw[1]}, {\tt uvw[2]}
  \end{enumerate}
\end{minipage}
\begin{minipage}[t]{8cm}
  \winfig{width=8cm}{hy_coldef.png}
\end{minipage}
\paragap

\subsection{Identify Hyades graphically in 3-d velocity space}

\begin{minipage}[t]{11cm}
  \raggedright
  \begin{enumerate}
  \item Plot points in 3-d space:
        \mb{plot2cube.png}{Graphics}{Cube Plot},
        \entry{X}{uvw[0]},
        \entry{Y}{uvw[1]},
        \entry{Z}{uvw[2]}.
        Note, you can type in any expression for the plot coordinates,
        you don't have to just select from available columns.
        The {\tt uvw} column itself doesn't appear in the selection list,
        since it's not a scalar.
  \item Select \entry{Mode}{\buttimg{MODE_DENSITY.png}density}
        in the \lab{Form} tab.  You can change the colour map to taste
        using the \lab{Density Shader} selector.
  \item Now, navigate through the cube to find an overdensity.
        This takes a bit of practice, but it's fun once you work out how.
        Click the little \buttimg{navig_help.png} button at bottom left
        for navigation help;
        the most useful actions are mouse wheel
        (2-fingered up/down drag on some trackpads) to zoom,
        and right click on a dense region to recenter.
  \item Navigate so only the objects in the overdense region
        (about 200 of them?) are visible inside the wireframe ---
        these are the Hyades.
  \item Use \mb{spoints5.png}{Subsets}{Subset From Visible},
        \entry{Name}{hyades}, \lab{Add Subset}.
  \end{enumerate}
\end{minipage}
\begin{minipage}[t]{8cm}
  \begin{picture}(8,0)
    \put(0,-9){\winfig{width=8cm}{hy_phase.png}}
  \end{picture}
\end{minipage}


\newpage
\subsection{View positions}

\begin{minipage}[t]{11cm}
  \begin{enumerate}
  \item Go back and plot this subset on the sky
        (\mb{skyplot_button.png}{Graphics}{Sky Plot})
  \item Use the \lab{Subsets} tab to make sure both \lab{hyades}
        and background objects (\lab{All}) are plotted in different colours.
        You might want to set \entry{Shading Mode}{flat} and
        fiddle with marker size and shape for clarity.
  \item To get an all-sky view,
        use the \buttimg{axes_button.png} \lab{Axes} control,
        \lab{Projection} tab, and change the \lab{Projection} selector
        from ``\lab{sin}'' to ``\lab{aitoff}''
  \item If you like, you can plot it in 3-d space as well:
        \mb{plot2sphere.png}{Graphics}{Sphere Plot},\\
        \entry{Lon}{ra},
        \entry{Lat}{dec},
        \entry{Radius}{1000./parallax}
  \end{enumerate}
\end{minipage}
\begin{minipage}[t]{8cm}
  \winfig{width=8cm}{hy_sky.png}
\end{minipage}

\subsection{Colour-magnitude diagram}

\begin{minipage}[t]{11cm}
  \raggedright
  \begin{enumerate}
  \item Plot a colour-magnitude diagram:
        \mb{planeplot_button.png}{Graphics}{Plane Plot}, \\
        \entry{X}{bp\_rp}, \entry{Y}{g\_abs}.
  \item Use the \buttimg{axes_button.png} \lab{Axes} control,
        \lab{Coords} tab, \lab{Y~Flip} checkbox
        to flip it the right way round.
  \item Use the \lab{Subsets} tab to make sure both \lab{hyades}
        and background objects (\lab{All}) are plotted in different colours.
        Hyades sit on a nice tight main sequence!
  \item There are a few outliers.
        Click on them, see the position show up in the sky plot too.
        In some cases, you can see by sky position that they are non-members.
        Open the \mb{browser1.png}{Views}{Data Window}
        and see that the relevant row is highlighted when
        you click on a plotted point.
  \end{enumerate}

  \vspace*{2cm}
  \subsection*{Bonus}
  \begin{itemize}
    \item Can you find any other clusters in velocity space?
    \item Try refining the selection by localising in position space too.
    \item What is the mean distance to the Hyades?
    \item Try using Aladin and SAMP along with TOPCAT
          to investigate the outliers.
  \end{itemize}
\end{minipage}
\begin{minipage}[t]{8cm}
  \winfig{width=8cm}{hy_cmd.png}
\end{minipage}

\newpage
\section{Match Gaia and HST observations}

In this example we have a local catalogue from a publication
by Gouliermis et al. 2006 \cite{gouliermis},
available in VizieR as J/ApJS/166/549.
This contains about 100\,000 sources observed by the ACS instrument
on the Hubble Space Telescope at epoch $\approx$ J2004.6
of stars in NGC346, a cluster in the Small Magellanic Cloud.
We match these positions with positions in the main Gaia catalogue
at J2016.0.

\subsection{Acquire HST observations}

\begin{minipage}[t]{11cm}
  \raggedright
  There are various ways to do this, but here
  we will use TOPCAT's VizieR dialogue window,
  which talks directly to the VizieR catalogue service.
  \begin{enumerate}
  \item Open \mb{vizier_button.png}{VO}{VizieR Catalogue Service} window
  \item \entry{Object Name}{ngc346}, and \lab{Resolve} to fill in
        \lab{RA} and \lab{Dec}
  \item \entry{Radius}{1} (degrees)
  \item \entry{Maximum Row Count}{100000} (or some large number)
  \item Catalogue selection panel: \lab{By Keyword} tab
  \item Fill in \entry{Keywords}{Gouliermis}
  \item Select ``{\tt J/ApJS/166/549}''
  \item Hit the \lab{OK} button at the bottom.
        A new table with 99\,079 rows should be loaded
  \end{enumerate}
  Alternatively, you could download the table from the VizieR web page.
\end{minipage}
\begin{minipage}[t]{8cm}
  \vspace*{-1cm}
  \winfig{width=8cm}{n3_vizier.png}
\end{minipage}


\subsection{Crossmatch with Gaia}
\label{sec:cdsxmatch}

\begin{minipage}[t]{11cm}
  \raggedright
  Now we want to find associations of the HST objects with sources
  from Gaia EDR3.
  Use the CDS X-Match service from TOPCAT.
  This uploads a local table to the CDS X-Match service,
  where the match is made against the Gaia EDR3 catalogue
  (or any other catalogue in VizieR).
  The resulting matched catalogue is then received as a new
  table in TOPCAT.
  \begin{enumerate}
  \item Open the \mb{xmatch_button.png}{VO}{CDS Upload X-Match} window
  \item Fill in the fields:\\
        \entry{VizieR Table ID/Alias}{GAIA EDR3} \\
        \entry{Input Table}{J\_ApJS\_166\_549\_table2} \\
        (or whatever the HST table is called) \\
        \entry{RA column}{\_RAJ000}, \entry{Dec column}{\_DEJ2000}  \\
        (should be filled in automatically) \\
        \entry{Radius}{1} (arcsec) \\
        \entry{Find Mode}{All} \hspace{1em}{\em Important!}
  \item Hit \lab{Go};
        within a few seconds, it should inform you that a new table has been
        loaded, with about 24\,000 rows.
  \item Look at the columns of the new table (all HST followed by all Gaia)
        in the \mb{cols_button.png}{Views}{Column Info} window
  \end{enumerate}
Note: the match is done with Gaia coordinates rolled back
(using Gaia proper motions) to J2000.0.
These propagated columns are in the matched table as
{\tt ra\_epoch2000}, {\tt dec\_epoch2000}.
\end{minipage}
\begin{minipage}[t]{8cm}
  \vspace*{-0.5cm}
  \winfig{width=8cm}{n3_cdsxmatch.png}
\end{minipage}


\subsection{Visualise the crossmatch}
\label{sec:xmatch-plot}

\begin{minipage}[t]{11cm}
  \raggedright
  \begin{enumerate}
  \item Open a \mb{skyplot_button.png}{Graphics}{Sky Plot} window
  \item Plot the HST observations:
        \entry{Table}{J\_ApJS\_166\_549\_table2},
        \entry{Lon}{\_RAJ2000}, \entry{Lat}{\_DEJ2000}
  \item Overplot the actual matches.
        Add a new {Pair} layer:
        \mb{pairs_button.png}{Layers}{Add Pair Control} and fill in:\\
        \entry{Table}{4xGAIA EDR3}
              (or whatever the xmatch result table is called)
        and both sets of coordinates: \\
        \entry{Lon(1)}{\_RAJ2000}, \entry{Lat(1)}{\_DEJ2000} (HST) \\
        \entry{Lon(2)}{ra}, \entry{Lat(2)}{dec} (Gaia)
  \item Zoom in to look at the associations.  There are too many!
        What is this plot telling you?
  \item You can fiddle around with the \lab{Form} tab to make the
        plot clearer, e.g. add a \ma{mark2_button.png}{Mark2} layer;
        change marker size, shape or colour.
  \end{enumerate}
  Visualising the results of a crossmatch is very often a good
  idea, unless you're pretty sure what you're going to get.
  Here, you can see it was crucial to understand the results:
  most of these matches are spurious, because there is a high density
  of HST sources.
\end{minipage}
\begin{minipage}[t]{8cm}
  \vspace*{-0.5cm}
  \winfig{width=8cm}{n3_allmatch.png}
\end{minipage}

\vspace*{-0.6cm}

\subsection{Investigate and identify matches}

\begin{minipage}[t]{11cm}
  \raggedright
  \begin{enumerate}
  \item Add new columns giving RA/Dec discrepancies between HST and Gaia
        positions: \\
        Open \mb{cols_button.png}{Views}{Column Info} window, \\
        then define new columns using
        \mb{plus_button.png}{Columns}{New Synthetic Column}: \\
        \entry{Name}{deltaRa}, \entry{Expression}{3600*(ra - \_RAJ2000)},
                               \entry{Units}{arcsec} \\
        \entry{Name}{deltaDec}, \entry{Expression}{3600*(dec - \_DEJ2000)},
                                \entry{Units}{arcsec}
  \item Use \mb{planeplot_button.png}{Graphics}{Plane Plot} window, \\
        plot \entry{X}{deltaRa}, \entry{Y}{deltaDec}
  \item Identify overdense region, select as in section~\ref{sec:blob},
        define new subset \lab{true\_match}.
  \item Go back to the sky associations plot from the previous section,
        and use the \lab{Subsets} tab to visualise which are the true matches.
        Why do you think this offset is not zero?
  \item Make a colour-colour diagram combining HST and Gaia photometry: \\
        Use \mb{planeplot_button.png}{Graphics}{Plane Plot} window,
        plot \entry{X}{Vmag-Imag}, \entry{Y}{bp\_rp},
        display \lab{true\_match} subset only. \\
        What are the two populations?
  \end{enumerate}
\end{minipage}
\begin{minipage}[t]{8cm}
  \winfig{width=8cm}{n3_truematch.png}
\end{minipage}

\paragap
Note: in the above we have used the Gaia columns {\tt ra} and {\tt dec},
giving the positions observed at J2016.0.
It would be a bit better to use {\tt ra\_epoch2000} and {\tt dec\_epoch2000},
the positions rolled back to J2000.0 using Gaia proper motions,
since the HST observations were at about J2014.6, and even better to
apply proper motions to get the positions at J2014.6.
But the differences (motion over 10--15 years) are fairly small.


\begin{minipage}[t]{11cm}

  \subsection{Alternative crossmatch: use local files}
  \raggedright
  The crossmatch in section \ref{sec:cdsxmatch} was done by sending
  a local file to an external service.
  This is often an efficient way to do it, but there are other options.
  Here, we will do the same crossmatch by operating on two local files
  with positions covering the same sky region.
  \begin{enumerate}
  \item First, retrieve Gaia data in the region of interest.
        Use the \mb{CONE_DIALOG.png}{VO}{Cone Search} window
        to a Gaia service as in
        section \ref{sec:m4-cone}, but this time fill in
        \entry{Object Name}{ngc346}, \entry{Radius}{0.05} (degrees).
  \item Plot the Gouliermis (from VizieR) and Gaia (from Cone Search)
        datasets on the sky:
        open the \mb{skyplot_button.png}{Graphics}{Sky Plot} window,
        fill in RA and Dec as \lab{Lat} and \lab{Lon} for one of the tables,
        then use the \mb{poscontrol_button.png}{Layers}{Add Position Control}
        action to overplot the same thing for the other dataset.
  \item Open the \mb{pairmatch_button.png}{Joins}{Pair Match} window
        from the main control window.
        Default \lab{Match Criteria} (Sky, 1 arcsec) are OK in this case.
        Fill in the \lab{Table 1} and \lab{Table 2} details for the
        Gouliermis and Gaia tables.
        Entry \entry{Match Selection}{All matches}.
  \item Hit \lab{Go} and wait a few seconds for the match to complete.
  \item When complete, a popup window will tell you, and offer to
        \ma{matchplot2.png}{Plot Result}.
        If you select this option, you will see a plot like the one from
        section \ref{sec:xmatch-plot}, except that the unmatched Gaia
        sources are also plotted (which can sometimes be useful information).
  \end{enumerate}
  Other matching options are available in the local match windows,
  including identifying objects that don't match, matching internally
  within a table, matching between three or more tables, etc.
  Note that unlike most things in TOPCAT, crossmatching can take up
  significant amounts of memory, so matching multi-million-row tables
  can sometimes grind to a halt or fail.

  \subsection*{Bonus}

  \begin{itemize}
  \item Use the \mb{histo_button.png}{Graphics}{Histogram Plot}
        (as in section \ref{sec:histo-mean})
        to find the mean values of $\delta$\,RA, $\delta$\,Dec for
        the true matches.  What do these values tell you?
  \item The match done here is between Gaia positions at J2016.0 and
        HST positions taken at approximately J2004.6.
        When using the X-Match service in section \ref{sec:cdsxmatch},
        Gaia proper motions were automatically applied
        to predict the Gaia positions at J2000.0.
        Use the {\tt epochProp} function in the expression language
        to do the match with the positions as Gaia proper motions
        predict for J2004.6.  Does it make much difference?
  \end{itemize}
\end{minipage}
\begin{minipage}[t]{8cm}
  \winfig{width=8cm}{n3_skyboth.png}
  \vspace*{1cm}

  \winfig{width=8cm}{n3_pairmatch.png}
\end{minipage}


\newpage
\section{Local Herzsprung-Russell Diagram}
\label{sec:hrd}

In this example we will use a TAP query to download
all the nearby Gaia sources with good astrometry and photometry,
and calculate their absolute magnitudes to construct an HR diagram,
performing a couple of cleaning operations to improve the data.
This procedure loosely follows
Appendix~C of the Gaia DR2 astrometry paper
Lindegren et al.\ 2018 \cite{lindegren}.

\subsection{Acquire data from TAP service}
\label{sec:hrd-tap}

\begin{minipage}[t]{11cm}
  \raggedright
  \begin{enumerate}
  \item Open the TAP window
        \mb{tap_button.png}{VO}{Table Access Protocol (TAP) Query}
  \item Select one of the Gaia services (probably the ESA one)
        and \lab{Use Service}
  \item Choose \entry{Mode}{Asynchronous}
        (just above the ADQL text entry panel).
        This query may take a minute or two,
        so a synchronous query might time out
        (with the unhelpful result, probably,
         ``{\em TAP response is not a VOTable}'').
  \item Execute the following query:
\label{step:hrd-data}
    \begin{verbatim}
    SELECT ra, dec, parallax, phot_g_mean_mag, bp_rp,
           astrometric_excess_noise,
           phot_bp_rp_excess_factor
    FROM gaiadr2.gaia_source
    WHERE parallax > 10
      AND parallax_over_error > 10
      AND phot_bp_mean_flux_over_error > 10
      AND phot_rp_mean_flux_over_error > 10
    \end{verbatim}
    \vspace*{-3ex}
    You should get a table with 338\,833 sources;
    they are nominally within 100\,pc, and have $\varpi$, BP and RP
    with small errors.  In particular, the parallax error is small enough
    that $\varpi^{-1}$ is a reasonable estimate of distance.
  \end{enumerate}
\end{minipage}

\subsection{Plot HRD}
\label{sec:plot-hrd}
\begin{minipage}[t]{11cm}
  \raggedright
  \begin{enumerate}
  \item Add a new column calculating absolute G magnitude, using parallax:
        \mb{plus_button.png}{Columns}{New Synthetic Column} in
        \ma{cols_button.png}{Column Info} window:\\
        \entry{Name}{g\_abs}, \\
        \entry{Expression}{phot\_g\_mean\_mag + 5*log10(parallax/100)}, \\
        \entry{Units}{mag}
        \label{item:calc-absmag}
  \item Make a \mb{planeplot_button.png}{Graphics}{Plane Plot}, with
        \entry{X}{bp\_rp} ($BP-RP$ colour), \entry{Y}{g\_abs}
        (absolute G magnitude).
        Use the \buttimg{axes_button.png} \lab{Axes} control,
        \lab{Coords} tab, \lab{Y~Flip} checkbox
        to flip it the right way round.
        Structure visible, but lots of interlopers.
  \item Play around with different \lab{Shading Modes} in \lab{Form} tab.
  \item Colour points using the other columns
        using modes \lab{Aux}, \lab{Weighted}.
        What's the difference between the two?
  \end{enumerate}
\end{minipage}
\begin{minipage}[t]{8cm}
  \vspace*{-8cm}
  \winfig{width=8cm}{hrd_excess.png}
\end{minipage}

\newpage
\subsection{Exclude astrometrically suspect sources}
\label{sec:hrd-astrom}

\begin{enumerate}
\item Try different weighting/aux columns to see which one can be used
      to exclude points in the unwanted region between the main
      sequence and white dwarf branch.
\item Experiment with the colour map settings
      (\ma{colours2.png}{Aux Axis} control)
      to find a suitable threshold.
\item Create a subset using an algebraic expression that excludes
      the spurious points:
      go to the \mb{subsets_button.png}{Views}{Row Subsets} window
      and use the \mb{plus_button.png}{Subsets}{New Subset} action: 
      \mbox{\entry{Subset Name}{astrom\_ok}},
      \mbox{\entry{Expression}{astrometric\_excess\_noise $<$ 1}}
\item Go back to the plot, and make sure only the {\tt astrom\_ok}
      subset is plotted (\lab{Subsets} tab).
      Now there are fewer spurious sources.
\end{enumerate}
The {\tt astrometric\_excess\_noise} column characterises the
goodness of fit of the astrometric solution to the observations.
But you didn't need to know that to improve the data like this.

\subsection{Exclude photometrically suspect sources}
\label{sec:hrd-photom}

\begin{minipage}[t]{11cm}
  \raggedright
  \begin{enumerate}
  \item Open a new \mb{planeplot_button.png}{Graphics}{Plane Plot}, with
        \entry{X}{bp\_rp} ($BP-RP$ colour),
        \entry{Y}{phot\_bp\_rp\_excess\_factor}.
  \end{enumerate}
  The quantity on the Y axis is some measure of photometric reliability.
  High values are bad, but how high is colour-dependent.
  We will define a region in this space to exclude the unusually high values.
  This time we will interactively draw a polygon rather than a blob.
  \begin{enumerate}
  \setcounter{enumi}{1}
  \item Click the \mb{poly2.png}{Subsets}{Draw Subset Polygon} action
  \item Select \entry{Point inclusion mode}{BELOW} in the popup
  \item Click on a few points above the overdense region until the shaded
        area roughly covers it.
  \item When you're done, click on the \buttimg{unpoly2.png} button again.
  \item Fill in the \lab{Subset Name} field in the popup
        (e.g. {\tt photom\_ok}) and hit \lab{OK}
  \item Go to the \mb{subsets_button.png}{Views}{Row Subsets} window,
        where you can see the new subset alongside {\tt astrom\_ok}.
  \item Create a new subset combining the two: \\
        \mb{plus_button.png}{Subsets}{New Subset} action: 
        \entry{Subset Name}{ok} \\
        \entry{Expression}{astrom\_ok \&\& photom\_ok}
  \item Go back to the plot, and make sure only the {\tt ok}
        subset is plotted (\lab{Subsets} tab).
        Now there are fewer spurious sources.
  \end{enumerate}
  Leave this session open, we'll need some of it for the next section.
\end{minipage}
\begin{minipage}[t]{8cm}
  \winfig{width=8cm}{hrd_photom.png}
\end{minipage}

\subsection{Explore the HRD}

\begin{minipage}[t]{11cm}
  \raggedright
  The photometry and astrometry in Gaia DR2 is so good that plotting
  a Herzsprung-Russell Diagram by following the steps above gives
  a lot of astrophysical information.
  Identify the different populations with the help of the zoom and
  shading options in TOPCAT: main sequence, giant branches,
  the double-stranded white dwarfs sequence representing the
  split between helium and hydrogen burning, and if you look
  closely a notch in the main sequence near absolute G magnitude of 10
  (Jao et al 2018 \cite{2018ApJ...861L..11J}).
\end{minipage}
\begin{minipage}[t]{8cm}
  \vspace*{-1.5cm}
  \winfig{width=8cm}{hrd_only.png}
\end{minipage}

\newpage

\section{Local Herzsprung-Russell Diagram using STILTS}

In this section we will reproduce the cleaned HRD from the previous section,
but this time from the command line, though we will use some of the results
of the earlier GUI activity.
Using TOPCAT interactively to work out how to write STILTS scripts
for later batch use like this is a common pattern of work.

\subsection{Get STILTS running}

\begin{enumerate}
\item Unlike TOPCAT, you can't bluff your way through STILTS
      by pointing'n'clicking; you'll need some documentation.
      Find the manual at \turl{http://www.starlink.ac.uk/stilts/sun256/},
      or just google for ``stilts'' and go to the \lab{Documentation} section.
\item Make sure you have it installed.  If you have topcat, you can
      run ``{\tt java -jar topcat-full.jar -stilts ...}''.
      More convenient (on Un*x), download the {\tt stilts} script
      into the same directory as {\tt topcat-full.jar} (or {\tt stilts.jar}).
      We will write just ``{\tt stilts}'' from now on.
\item Run a command:
      \vspace*{-2ex}
      \begin{verbatim}
      stilts calc expression="1+2"
      \end{verbatim}
      \vspace*{-4ex}
      should print out ``{\tt 3}''
\item Find the documentation for the {\tt calc} command in the manual
      (Appendix B).
      Look at the \lab{Usage} and \lab{Examples} subsections.
\item Get command-line help by running ``{\tt stilts calc help}'' and
      ``{\tt stilts calc help=expression}''.
\end{enumerate}

\subsection{Acquire data from TAP service}

\begin{enumerate}
\item We will use the {\tt tapquery} command;
      find the entry in the manual.
\item Go back to the TOPCAT TAP window to find the TAP URL at the
      bottom of the \lab{Select Service} tab (value of {\tt tapurl} parameter)
\item Use the ADQL from step \ref{step:hrd-data} of section \ref{sec:hrd-tap}
      (value of {\tt adql} parameter).
\item Run the query and output the result to a local FITS file {\tt hrd.fits}:
      \begin{verbatim}
          stilts tapquery \
            sync=false \
            tapurl=http://gea.esac.esa.int/tap-server/tap \
            adql="SELECT ra, dec, parallax, phot_g_mean_mag, \
                         phot_g_mean_mag, bp_rp \
                         astrometric_excess_noise, \
                         phot_bp_rp_excess_factor \
                  FROM gaiadr2.gaia_source \
                  WHERE parallax > 10 \
                    AND parallax_over_error > 10 \
                    AND phot_bp_mean_flux_over_error > 10 \
                    AND phot_rp_mean_flux_over_error > 10" \
            out=hrd.fits
      \end{verbatim}
\end{enumerate}

\subsection{Manipulate the downloaded table using {\tt tpipe}}

We will perform some processing on the downloaded table using
the {\tt tpipe} command.
This reads an input table, optionally performs operations on it,
and writes it to output, either to a file (in the same or different format)
or some other destination.  It works like a Unix pipeline.

The operations are defined by adding {\em filters} such as
``{\tt select}'' (retain only some rows) or ``{\tt addcol}''
(add a new column).  Filters are specified by adding using zero
or more ``{\tt cmd=}'' parameters on the command line,
and documented in Section 6.1: ``Processing Filters'' of the STILTS manual.

The default output mode is write to a file,
but other options such as ``{\tt meta}'' (display column metadata),
``{\tt count}'' (count rows) and ``{\tt stats}'' (calculate mean, st.dev etc)
are also available.
Select non-default output modes using the ``{\tt omode=}'' parameter
on the command line, documented in Section 6.4: ``Output Modes'' of the manual.

\begin{enumerate}
\item Count the rows in the query result table:
      \begin{verbatim}
      stilts tpipe in=hrd.fits omode=count
      \end{verbatim}
      \vspace*{-3ex}
\item See what columns the query result table has:
      \begin{verbatim}
      stilts tpipe in=hrd.fits omode=meta
      \end{verbatim}
      \vspace*{-3ex}
\item Prepare an expression for a new column giving absolute magnitude
      (see item \ref{item:calc-absmag} from section \ref{sec:plot-hrd})
      \label{item:hrd-addcol}
\item Prepare expressions for rows corresponding to the
      astrometry and photometry selections
      from sections \ref{sec:hrd-astrom} and \ref{sec:hrd-photom};
      go back to TOPCAT's \lab{Subsets} window, and see the expressions
      used to restrict the sources in the final HRD.
      \label{item:hrd-select}
\item Put items \ref{item:hrd-addcol} and \ref{item:hrd-select} together
      to produce a plot-ready table (note use of quotes):
      \begin{verbatim}
  stilts tpipe in=hrd.fits \
               cmd='addcol g_abs "phot_g_mean_mag + 5*log10(parallax/100)"' \
               cmd='select "astrometric_excess_noise < 1"' \
               cmd='select "phot_bp_rp_excess_factor < polyLine(bp_rp, -0.56,1.307, 0.03,1.192, \
                                                                       1.51,1.295, 4.31,1.808)"' \
               out=hrd_clean.fits
      \end{verbatim}
      \vspace*{-3ex}
\item Count the rows in the output file.
\end{enumerate}

\begin{minipage}[t]{11cm}
  \raggedright
  \vspace*{1ex}
  \subsection{Plot the diagram}

  \begin{enumerate}
  \item Plot the HRD on the screen using
        the 2d plotting command {\tt plot2plane}:
  \vspace*{-1ex}
  \begin{verbatim}
     stilts plot2plane layer=mark in=hrd_clean.fits \
                       x=bp_rp y=g_abs yflip=true
  \end{verbatim}
  \vspace*{-4ex}
  \item By default, a ``live'' plot shows up on the screen:
        use the mouse to zoom and drag it around like in TOPCAT.
  \item Output it to a graphics file instead,
        by adding the ``{\tt out}'' parameter,
        e.g.\ ``{\tt out=hrd.png}'' or ``{\tt out=hrd.pdf}''.
  \end{enumerate}
\end{minipage}
\begin{minipage}[t]{8cm}
  \begin{flushright}
  \vspace*{-0.6cm}
  \winfig{width=6cm}{hrd_plot2plane.png}
  \end{flushright}
\end{minipage}

\begin{minipage}[t]{11cm}
  \vspace*{2cm}
  \subsection{Use TOPCAT to refine the plot command}
  There are {\em lots\/} more options to the plot commands.

  \begin{enumerate}
  \item In the cleaned HRD plot from the end of section~\ref{sec:hrd},
        select the \ma{stilts.png}{STILTS} control at the lower left.
        This is the command that will reproduce the plot from the command line.
        It probably won't work from the command line exactly as written,
        because some tables and subsets only exist in TOPCAT
        (expected problems are shown in colour).
        But we have now generated the post-processed file using {\tt tpipe}.
  \item Hit the \lab{Window} button near the bottom to break out
        the command display into a separate window.
        You can see {\em all\/} the options (including defaults) with the
        \lab{Formatting}$\mid$\lab{Include Defaults} menu item.
  \item Now adjust plot options in the TOPCAT GUI and see how the command
        changes.
  \item Get the plot looking like you want in TOPCAT, then copy and paste
        the details into your STILTS {\tt plot2plane} command line.
  \item Experiment with changing the appearance by modifying the script.
  \end{enumerate}
\end{minipage}
\begin{minipage}[t]{8cm}
  \winfig{width=8cm}{hrd_stilts.png}
\end{minipage}

% Use the expression from section \ref{sec:hrd-stilts-plot},
% but replace the ``{\tt in}'' and ``{\tt icmd}'' values with
% ``{\tt in=hrd_clean.fits}'':

\newpage


\section*{Bonus}

Go back and reproduce all the other exercises using STILTS!



\begin{thebibliography}{9}
\bibitem{edr3}
   Gaia Collaboration et al.,
   ``Gaia Early Data Release 3: Summary of the contents and survey properties'',
   Astronomy and Astrophysics
   {\tt https://doi.org/10.1051/0004-6361/202039657} (2021)
\bibitem{luri}
   X.Luri et al.,
   ``Gaia DR2: Using Gaia parallaxes'',
   Astronomy and Astrophysics {\em 616}, A9 (2018),
   {\tt 2018A\&A...616A...9L}
\bibitem{lindegren}
   L.Lindegren et al.,
   ``Gaia DR2: The astrometric solution'',
   Astronomy and Astrophysics {\em 616}, A2 (2018),
   {\tt 2018A\&A...616A...2L}
\bibitem{gouliermis}
   D.A.Gouliermis, A.E.Dolphin, W.Brandner and Th.Henning,
   ``The Star-forming Region NGC 346 in the Small Magellanic
     Cloud with Hubble Space Telescope ACS Observations. I. Photometry'',
   ApJS, 166 p.549
   {\tt 2006ApJS..166..549G}
\bibitem{2018ApJ...861L..11J}
   W.-C.Jao, T.J.Henry, D.R.Gies, N.C.Hambly,
   ``A Gap in the Lower Main Sequence Revealed by Gaia Data Release 2'',
   ApJ Letts, 861, L11 (2018),
   {\tt 2018ApJ...861L..11J}
\end{thebibliography}

\section*{Acknowledgements}

This work has made use of data from the European Space Agency (ESA) mission
{\it Gaia} (\url{https://www.cosmos.esa.int/gaia}), processed by the {\it Gaia}
Data Processing and Analysis Consortium
(DPAC, \url{https://www.cosmos.esa.int/web/gaia/dpac/consortium}).
Funding for the DPAC
has been provided by national institutions, in particular the institutions
participating in the {\it Gaia} Multilateral Agreement.


\label{lastPage}


\end{document}

